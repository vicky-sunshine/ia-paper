\section{Introduction}
% Network function 重要性
In these days, the network functions have took its place in modern life, play an important role in providing critical performance, security and policy compliance capabilities. However, network service is usually achieved by the hardware and using a dedicated device to deploy and integrate a variety of network functions, which causes the lack of flexibility in service delivery. When the network service provider wants to introduce the new service, customers will tend not to replace with new equipment and the service provider must also cost high capital expenses (CapEx) and operation expenses (OpEx) to replace the physical equipment. As a result, the network service is difficult to promote and update.

% NFV show up
It was against this background, the concept of network function virtualization (NFV) was put forward, expecting to solve these problem. NFV transforms how network operators architect their infrastructures with the virtualization technology to separate software instance from hardware platform, and implements virtualized network functions (VNFs) through software techniques. And NFV also innovates in the service delivery, by deploying these functions over general virtualized compute infrastructures, which are used to replace the traditional network appliances. When these functions are no longer a large degree depent on specific physical devices, the management of VNFs can be easier and the deployment will achieve flexibility and scalability.

% SDN help NFV -> SDN-enabled NFV, which NFVI is part of function
With Software Defined Network, which represents the idea and technology of decoupling the control plane from the underlying data plane with, the evolution virtualization of the network functions is a big step forward.  At the beginning, SDN simply contribute the network infrastructure with enabling dynamic control and configuration of network nodes. Afterwards, the network infrastructure layer become an implementation part of the VNF functionality.

% SDN is benefit for espectailly CPE provider
This concept of SDN-enabled VNF fits perfectly the needs of service providers, especially providers of Customer Premises Equipment (CPE). In the current physical Consumer Premises Equipment (pCPE) model, service providers have to deploy multiple discrete devices. The range of services region are diverse, which make installation taking long time and high CAPEX, and when service providers want to promote new services, the customers have low aspiration to purchase

% advantage of vCPE implemented with SDN-enabled architecture
Though the technology of SDN-enabled VNF, the concept of virtual Consumer Premises Equipment (vCPE) has been discussed. With vCPE platform, service providers will be able to provide services through internet and the customer may need to buy only one low-cost device. As a result, it will shorten installation time, reduce the maintenance cost of pCPE, and increasing the purchase intention of customers.

% this paper
In this paper, we proposed a vCPE framework architecture to provide SDN-enabled NFV service. The proposed framework is especially suitable for service provider to deliver their service. Consequently, there is a great demand to use the vCPE framework to lower the devices cost as well as maintenance cost.

% outline
The rest of the paper is organized as follows: Some related work will be
described in Section 2. The proposed vCPE system architecture and
implementation details are presented in Section 3. The section 4 introduced some SDN-enabled VNF we have implemented. The Section 5 depicts the experimental environment that verifies the proposed architecture
and mechanisms. Lastly, the conclusion and future works are presented
in Section 6.
